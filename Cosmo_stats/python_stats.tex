
% Default to the notebook output style

    


% Inherit from the specified cell style.




    
\documentclass[11pt]{article}

    
    
    \usepackage[T1]{fontenc}
    % Nicer default font (+ math font) than Computer Modern for most use cases
    \usepackage{mathpazo}

    % Basic figure setup, for now with no caption control since it's done
    % automatically by Pandoc (which extracts ![](path) syntax from Markdown).
    \usepackage{graphicx}
    % We will generate all images so they have a width \maxwidth. This means
    % that they will get their normal width if they fit onto the page, but
    % are scaled down if they would overflow the margins.
    \makeatletter
    \def\maxwidth{\ifdim\Gin@nat@width>\linewidth\linewidth
    \else\Gin@nat@width\fi}
    \makeatother
    \let\Oldincludegraphics\includegraphics
    % Set max figure width to be 80% of text width, for now hardcoded.
    \renewcommand{\includegraphics}[1]{\Oldincludegraphics[width=.8\maxwidth]{#1}}
    % Ensure that by default, figures have no caption (until we provide a
    % proper Figure object with a Caption API and a way to capture that
    % in the conversion process - todo).
    \usepackage{caption}
    \DeclareCaptionLabelFormat{nolabel}{}
    \captionsetup{labelformat=nolabel}

    \usepackage{adjustbox} % Used to constrain images to a maximum size 
    \usepackage{xcolor} % Allow colors to be defined
    \usepackage{enumerate} % Needed for markdown enumerations to work
    \usepackage{geometry} % Used to adjust the document margins
    \usepackage{amsmath} % Equations
    \usepackage{amssymb} % Equations
    \usepackage{textcomp} % defines textquotesingle
    % Hack from http://tex.stackexchange.com/a/47451/13684:
    \AtBeginDocument{%
        \def\PYZsq{\textquotesingle}% Upright quotes in Pygmentized code
    }
    \usepackage{upquote} % Upright quotes for verbatim code
    \usepackage{eurosym} % defines \euro
    \usepackage[mathletters]{ucs} % Extended unicode (utf-8) support
    \usepackage[utf8x]{inputenc} % Allow utf-8 characters in the tex document
    \usepackage{fancyvrb} % verbatim replacement that allows latex
    \usepackage{grffile} % extends the file name processing of package graphics 
                         % to support a larger range 
    % The hyperref package gives us a pdf with properly built
    % internal navigation ('pdf bookmarks' for the table of contents,
    % internal cross-reference links, web links for URLs, etc.)
    \usepackage{hyperref}
    \usepackage{longtable} % longtable support required by pandoc >1.10
    \usepackage{booktabs}  % table support for pandoc > 1.12.2
    \usepackage[inline]{enumitem} % IRkernel/repr support (it uses the enumerate* environment)
    \usepackage[normalem]{ulem} % ulem is needed to support strikethroughs (\sout)
                                % normalem makes italics be italics, not underlines
    

    
    
    % Colors for the hyperref package
    \definecolor{urlcolor}{rgb}{0,.145,.698}
    \definecolor{linkcolor}{rgb}{.71,0.21,0.01}
    \definecolor{citecolor}{rgb}{.12,.54,.11}

    % ANSI colors
    \definecolor{ansi-black}{HTML}{3E424D}
    \definecolor{ansi-black-intense}{HTML}{282C36}
    \definecolor{ansi-red}{HTML}{E75C58}
    \definecolor{ansi-red-intense}{HTML}{B22B31}
    \definecolor{ansi-green}{HTML}{00A250}
    \definecolor{ansi-green-intense}{HTML}{007427}
    \definecolor{ansi-yellow}{HTML}{DDB62B}
    \definecolor{ansi-yellow-intense}{HTML}{B27D12}
    \definecolor{ansi-blue}{HTML}{208FFB}
    \definecolor{ansi-blue-intense}{HTML}{0065CA}
    \definecolor{ansi-magenta}{HTML}{D160C4}
    \definecolor{ansi-magenta-intense}{HTML}{A03196}
    \definecolor{ansi-cyan}{HTML}{60C6C8}
    \definecolor{ansi-cyan-intense}{HTML}{258F8F}
    \definecolor{ansi-white}{HTML}{C5C1B4}
    \definecolor{ansi-white-intense}{HTML}{A1A6B2}

    % commands and environments needed by pandoc snippets
    % extracted from the output of `pandoc -s`
    \providecommand{\tightlist}{%
      \setlength{\itemsep}{0pt}\setlength{\parskip}{0pt}}
    \DefineVerbatimEnvironment{Highlighting}{Verbatim}{commandchars=\\\{\}}
    % Add ',fontsize=\small' for more characters per line
    \newenvironment{Shaded}{}{}
    \newcommand{\KeywordTok}[1]{\textcolor[rgb]{0.00,0.44,0.13}{\textbf{{#1}}}}
    \newcommand{\DataTypeTok}[1]{\textcolor[rgb]{0.56,0.13,0.00}{{#1}}}
    \newcommand{\DecValTok}[1]{\textcolor[rgb]{0.25,0.63,0.44}{{#1}}}
    \newcommand{\BaseNTok}[1]{\textcolor[rgb]{0.25,0.63,0.44}{{#1}}}
    \newcommand{\FloatTok}[1]{\textcolor[rgb]{0.25,0.63,0.44}{{#1}}}
    \newcommand{\CharTok}[1]{\textcolor[rgb]{0.25,0.44,0.63}{{#1}}}
    \newcommand{\StringTok}[1]{\textcolor[rgb]{0.25,0.44,0.63}{{#1}}}
    \newcommand{\CommentTok}[1]{\textcolor[rgb]{0.38,0.63,0.69}{\textit{{#1}}}}
    \newcommand{\OtherTok}[1]{\textcolor[rgb]{0.00,0.44,0.13}{{#1}}}
    \newcommand{\AlertTok}[1]{\textcolor[rgb]{1.00,0.00,0.00}{\textbf{{#1}}}}
    \newcommand{\FunctionTok}[1]{\textcolor[rgb]{0.02,0.16,0.49}{{#1}}}
    \newcommand{\RegionMarkerTok}[1]{{#1}}
    \newcommand{\ErrorTok}[1]{\textcolor[rgb]{1.00,0.00,0.00}{\textbf{{#1}}}}
    \newcommand{\NormalTok}[1]{{#1}}
    
    % Additional commands for more recent versions of Pandoc
    \newcommand{\ConstantTok}[1]{\textcolor[rgb]{0.53,0.00,0.00}{{#1}}}
    \newcommand{\SpecialCharTok}[1]{\textcolor[rgb]{0.25,0.44,0.63}{{#1}}}
    \newcommand{\VerbatimStringTok}[1]{\textcolor[rgb]{0.25,0.44,0.63}{{#1}}}
    \newcommand{\SpecialStringTok}[1]{\textcolor[rgb]{0.73,0.40,0.53}{{#1}}}
    \newcommand{\ImportTok}[1]{{#1}}
    \newcommand{\DocumentationTok}[1]{\textcolor[rgb]{0.73,0.13,0.13}{\textit{{#1}}}}
    \newcommand{\AnnotationTok}[1]{\textcolor[rgb]{0.38,0.63,0.69}{\textbf{\textit{{#1}}}}}
    \newcommand{\CommentVarTok}[1]{\textcolor[rgb]{0.38,0.63,0.69}{\textbf{\textit{{#1}}}}}
    \newcommand{\VariableTok}[1]{\textcolor[rgb]{0.10,0.09,0.49}{{#1}}}
    \newcommand{\ControlFlowTok}[1]{\textcolor[rgb]{0.00,0.44,0.13}{\textbf{{#1}}}}
    \newcommand{\OperatorTok}[1]{\textcolor[rgb]{0.40,0.40,0.40}{{#1}}}
    \newcommand{\BuiltInTok}[1]{{#1}}
    \newcommand{\ExtensionTok}[1]{{#1}}
    \newcommand{\PreprocessorTok}[1]{\textcolor[rgb]{0.74,0.48,0.00}{{#1}}}
    \newcommand{\AttributeTok}[1]{\textcolor[rgb]{0.49,0.56,0.16}{{#1}}}
    \newcommand{\InformationTok}[1]{\textcolor[rgb]{0.38,0.63,0.69}{\textbf{\textit{{#1}}}}}
    \newcommand{\WarningTok}[1]{\textcolor[rgb]{0.38,0.63,0.69}{\textbf{\textit{{#1}}}}}
    
    
    % Define a nice break command that doesn't care if a line doesn't already
    % exist.
    \def\br{\hspace*{\fill} \\* }
    % Math Jax compatability definitions
    \def\gt{>}
    \def\lt{<}
    % Document parameters
    \title{Untitled}
    
    
    

    % Pygments definitions
    
\makeatletter
\def\PY@reset{\let\PY@it=\relax \let\PY@bf=\relax%
    \let\PY@ul=\relax \let\PY@tc=\relax%
    \let\PY@bc=\relax \let\PY@ff=\relax}
\def\PY@tok#1{\csname PY@tok@#1\endcsname}
\def\PY@toks#1+{\ifx\relax#1\empty\else%
    \PY@tok{#1}\expandafter\PY@toks\fi}
\def\PY@do#1{\PY@bc{\PY@tc{\PY@ul{%
    \PY@it{\PY@bf{\PY@ff{#1}}}}}}}
\def\PY#1#2{\PY@reset\PY@toks#1+\relax+\PY@do{#2}}

\expandafter\def\csname PY@tok@gd\endcsname{\def\PY@tc##1{\textcolor[rgb]{0.63,0.00,0.00}{##1}}}
\expandafter\def\csname PY@tok@gu\endcsname{\let\PY@bf=\textbf\def\PY@tc##1{\textcolor[rgb]{0.50,0.00,0.50}{##1}}}
\expandafter\def\csname PY@tok@gt\endcsname{\def\PY@tc##1{\textcolor[rgb]{0.00,0.27,0.87}{##1}}}
\expandafter\def\csname PY@tok@gs\endcsname{\let\PY@bf=\textbf}
\expandafter\def\csname PY@tok@gr\endcsname{\def\PY@tc##1{\textcolor[rgb]{1.00,0.00,0.00}{##1}}}
\expandafter\def\csname PY@tok@cm\endcsname{\let\PY@it=\textit\def\PY@tc##1{\textcolor[rgb]{0.25,0.50,0.50}{##1}}}
\expandafter\def\csname PY@tok@vg\endcsname{\def\PY@tc##1{\textcolor[rgb]{0.10,0.09,0.49}{##1}}}
\expandafter\def\csname PY@tok@vi\endcsname{\def\PY@tc##1{\textcolor[rgb]{0.10,0.09,0.49}{##1}}}
\expandafter\def\csname PY@tok@vm\endcsname{\def\PY@tc##1{\textcolor[rgb]{0.10,0.09,0.49}{##1}}}
\expandafter\def\csname PY@tok@mh\endcsname{\def\PY@tc##1{\textcolor[rgb]{0.40,0.40,0.40}{##1}}}
\expandafter\def\csname PY@tok@cs\endcsname{\let\PY@it=\textit\def\PY@tc##1{\textcolor[rgb]{0.25,0.50,0.50}{##1}}}
\expandafter\def\csname PY@tok@ge\endcsname{\let\PY@it=\textit}
\expandafter\def\csname PY@tok@vc\endcsname{\def\PY@tc##1{\textcolor[rgb]{0.10,0.09,0.49}{##1}}}
\expandafter\def\csname PY@tok@il\endcsname{\def\PY@tc##1{\textcolor[rgb]{0.40,0.40,0.40}{##1}}}
\expandafter\def\csname PY@tok@go\endcsname{\def\PY@tc##1{\textcolor[rgb]{0.53,0.53,0.53}{##1}}}
\expandafter\def\csname PY@tok@cp\endcsname{\def\PY@tc##1{\textcolor[rgb]{0.74,0.48,0.00}{##1}}}
\expandafter\def\csname PY@tok@gi\endcsname{\def\PY@tc##1{\textcolor[rgb]{0.00,0.63,0.00}{##1}}}
\expandafter\def\csname PY@tok@gh\endcsname{\let\PY@bf=\textbf\def\PY@tc##1{\textcolor[rgb]{0.00,0.00,0.50}{##1}}}
\expandafter\def\csname PY@tok@ni\endcsname{\let\PY@bf=\textbf\def\PY@tc##1{\textcolor[rgb]{0.60,0.60,0.60}{##1}}}
\expandafter\def\csname PY@tok@nl\endcsname{\def\PY@tc##1{\textcolor[rgb]{0.63,0.63,0.00}{##1}}}
\expandafter\def\csname PY@tok@nn\endcsname{\let\PY@bf=\textbf\def\PY@tc##1{\textcolor[rgb]{0.00,0.00,1.00}{##1}}}
\expandafter\def\csname PY@tok@no\endcsname{\def\PY@tc##1{\textcolor[rgb]{0.53,0.00,0.00}{##1}}}
\expandafter\def\csname PY@tok@na\endcsname{\def\PY@tc##1{\textcolor[rgb]{0.49,0.56,0.16}{##1}}}
\expandafter\def\csname PY@tok@nb\endcsname{\def\PY@tc##1{\textcolor[rgb]{0.00,0.50,0.00}{##1}}}
\expandafter\def\csname PY@tok@nc\endcsname{\let\PY@bf=\textbf\def\PY@tc##1{\textcolor[rgb]{0.00,0.00,1.00}{##1}}}
\expandafter\def\csname PY@tok@nd\endcsname{\def\PY@tc##1{\textcolor[rgb]{0.67,0.13,1.00}{##1}}}
\expandafter\def\csname PY@tok@ne\endcsname{\let\PY@bf=\textbf\def\PY@tc##1{\textcolor[rgb]{0.82,0.25,0.23}{##1}}}
\expandafter\def\csname PY@tok@nf\endcsname{\def\PY@tc##1{\textcolor[rgb]{0.00,0.00,1.00}{##1}}}
\expandafter\def\csname PY@tok@si\endcsname{\let\PY@bf=\textbf\def\PY@tc##1{\textcolor[rgb]{0.73,0.40,0.53}{##1}}}
\expandafter\def\csname PY@tok@s2\endcsname{\def\PY@tc##1{\textcolor[rgb]{0.73,0.13,0.13}{##1}}}
\expandafter\def\csname PY@tok@nt\endcsname{\let\PY@bf=\textbf\def\PY@tc##1{\textcolor[rgb]{0.00,0.50,0.00}{##1}}}
\expandafter\def\csname PY@tok@nv\endcsname{\def\PY@tc##1{\textcolor[rgb]{0.10,0.09,0.49}{##1}}}
\expandafter\def\csname PY@tok@s1\endcsname{\def\PY@tc##1{\textcolor[rgb]{0.73,0.13,0.13}{##1}}}
\expandafter\def\csname PY@tok@dl\endcsname{\def\PY@tc##1{\textcolor[rgb]{0.73,0.13,0.13}{##1}}}
\expandafter\def\csname PY@tok@ch\endcsname{\let\PY@it=\textit\def\PY@tc##1{\textcolor[rgb]{0.25,0.50,0.50}{##1}}}
\expandafter\def\csname PY@tok@m\endcsname{\def\PY@tc##1{\textcolor[rgb]{0.40,0.40,0.40}{##1}}}
\expandafter\def\csname PY@tok@gp\endcsname{\let\PY@bf=\textbf\def\PY@tc##1{\textcolor[rgb]{0.00,0.00,0.50}{##1}}}
\expandafter\def\csname PY@tok@sh\endcsname{\def\PY@tc##1{\textcolor[rgb]{0.73,0.13,0.13}{##1}}}
\expandafter\def\csname PY@tok@ow\endcsname{\let\PY@bf=\textbf\def\PY@tc##1{\textcolor[rgb]{0.67,0.13,1.00}{##1}}}
\expandafter\def\csname PY@tok@sx\endcsname{\def\PY@tc##1{\textcolor[rgb]{0.00,0.50,0.00}{##1}}}
\expandafter\def\csname PY@tok@bp\endcsname{\def\PY@tc##1{\textcolor[rgb]{0.00,0.50,0.00}{##1}}}
\expandafter\def\csname PY@tok@c1\endcsname{\let\PY@it=\textit\def\PY@tc##1{\textcolor[rgb]{0.25,0.50,0.50}{##1}}}
\expandafter\def\csname PY@tok@fm\endcsname{\def\PY@tc##1{\textcolor[rgb]{0.00,0.00,1.00}{##1}}}
\expandafter\def\csname PY@tok@o\endcsname{\def\PY@tc##1{\textcolor[rgb]{0.40,0.40,0.40}{##1}}}
\expandafter\def\csname PY@tok@kc\endcsname{\let\PY@bf=\textbf\def\PY@tc##1{\textcolor[rgb]{0.00,0.50,0.00}{##1}}}
\expandafter\def\csname PY@tok@c\endcsname{\let\PY@it=\textit\def\PY@tc##1{\textcolor[rgb]{0.25,0.50,0.50}{##1}}}
\expandafter\def\csname PY@tok@mf\endcsname{\def\PY@tc##1{\textcolor[rgb]{0.40,0.40,0.40}{##1}}}
\expandafter\def\csname PY@tok@err\endcsname{\def\PY@bc##1{\setlength{\fboxsep}{0pt}\fcolorbox[rgb]{1.00,0.00,0.00}{1,1,1}{\strut ##1}}}
\expandafter\def\csname PY@tok@mb\endcsname{\def\PY@tc##1{\textcolor[rgb]{0.40,0.40,0.40}{##1}}}
\expandafter\def\csname PY@tok@ss\endcsname{\def\PY@tc##1{\textcolor[rgb]{0.10,0.09,0.49}{##1}}}
\expandafter\def\csname PY@tok@sr\endcsname{\def\PY@tc##1{\textcolor[rgb]{0.73,0.40,0.53}{##1}}}
\expandafter\def\csname PY@tok@mo\endcsname{\def\PY@tc##1{\textcolor[rgb]{0.40,0.40,0.40}{##1}}}
\expandafter\def\csname PY@tok@kd\endcsname{\let\PY@bf=\textbf\def\PY@tc##1{\textcolor[rgb]{0.00,0.50,0.00}{##1}}}
\expandafter\def\csname PY@tok@mi\endcsname{\def\PY@tc##1{\textcolor[rgb]{0.40,0.40,0.40}{##1}}}
\expandafter\def\csname PY@tok@kn\endcsname{\let\PY@bf=\textbf\def\PY@tc##1{\textcolor[rgb]{0.00,0.50,0.00}{##1}}}
\expandafter\def\csname PY@tok@cpf\endcsname{\let\PY@it=\textit\def\PY@tc##1{\textcolor[rgb]{0.25,0.50,0.50}{##1}}}
\expandafter\def\csname PY@tok@kr\endcsname{\let\PY@bf=\textbf\def\PY@tc##1{\textcolor[rgb]{0.00,0.50,0.00}{##1}}}
\expandafter\def\csname PY@tok@s\endcsname{\def\PY@tc##1{\textcolor[rgb]{0.73,0.13,0.13}{##1}}}
\expandafter\def\csname PY@tok@kp\endcsname{\def\PY@tc##1{\textcolor[rgb]{0.00,0.50,0.00}{##1}}}
\expandafter\def\csname PY@tok@w\endcsname{\def\PY@tc##1{\textcolor[rgb]{0.73,0.73,0.73}{##1}}}
\expandafter\def\csname PY@tok@kt\endcsname{\def\PY@tc##1{\textcolor[rgb]{0.69,0.00,0.25}{##1}}}
\expandafter\def\csname PY@tok@sc\endcsname{\def\PY@tc##1{\textcolor[rgb]{0.73,0.13,0.13}{##1}}}
\expandafter\def\csname PY@tok@sb\endcsname{\def\PY@tc##1{\textcolor[rgb]{0.73,0.13,0.13}{##1}}}
\expandafter\def\csname PY@tok@sa\endcsname{\def\PY@tc##1{\textcolor[rgb]{0.73,0.13,0.13}{##1}}}
\expandafter\def\csname PY@tok@k\endcsname{\let\PY@bf=\textbf\def\PY@tc##1{\textcolor[rgb]{0.00,0.50,0.00}{##1}}}
\expandafter\def\csname PY@tok@se\endcsname{\let\PY@bf=\textbf\def\PY@tc##1{\textcolor[rgb]{0.73,0.40,0.13}{##1}}}
\expandafter\def\csname PY@tok@sd\endcsname{\let\PY@it=\textit\def\PY@tc##1{\textcolor[rgb]{0.73,0.13,0.13}{##1}}}

\def\PYZbs{\char`\\}
\def\PYZus{\char`\_}
\def\PYZob{\char`\{}
\def\PYZcb{\char`\}}
\def\PYZca{\char`\^}
\def\PYZam{\char`\&}
\def\PYZlt{\char`\<}
\def\PYZgt{\char`\>}
\def\PYZsh{\char`\#}
\def\PYZpc{\char`\%}
\def\PYZdl{\char`\$}
\def\PYZhy{\char`\-}
\def\PYZsq{\char`\'}
\def\PYZdq{\char`\"}
\def\PYZti{\char`\~}
% for compatibility with earlier versions
\def\PYZat{@}
\def\PYZlb{[}
\def\PYZrb{]}
\makeatother


    % Exact colors from NB
    \definecolor{incolor}{rgb}{0.0, 0.0, 0.5}
    \definecolor{outcolor}{rgb}{0.545, 0.0, 0.0}



    
    % Prevent overflowing lines due to hard-to-break entities
    \sloppy 
    % Setup hyperref package
    \hypersetup{
      breaklinks=true,  % so long urls are correctly broken across lines
      colorlinks=true,
      urlcolor=urlcolor,
      linkcolor=linkcolor,
      citecolor=citecolor,
      }
    % Slightly bigger margins than the latex defaults
    
    \geometry{verbose,tmargin=1in,bmargin=1in,lmargin=1in,rmargin=1in}
    
    

    \begin{document}
    
    
    \maketitle
    
    

    
    \begin{Verbatim}[commandchars=\\\{\}]
{\color{incolor}In [{\color{incolor}111}]:} \PY{o}{\PYZpc{}}\PY{k}{pylab} inline
          
          \PY{k+kn}{import} \PY{n+nn}{pymc3} \PY{k+kn}{as} \PY{n+nn}{pm}
          \PY{k+kn}{from} \PY{n+nn}{pymc3.backends} \PY{k+kn}{import} \PY{n}{SQLite}
          \PY{k+kn}{import} \PY{n+nn}{triangle}
          \PY{k+kn}{import} \PY{n+nn}{pandas} \PY{k+kn}{as} \PY{n+nn}{pd}
\end{Verbatim}


    \begin{Verbatim}[commandchars=\\\{\}]
Populating the interactive namespace from numpy and matplotlib

    \end{Verbatim}

    \section{Los datos}\label{los-datos}

    \begin{Verbatim}[commandchars=\\\{\}]
{\color{incolor}In [{\color{incolor}112}]:} \PY{n}{dataHz} \PY{o}{=} \PY{n}{np}\PY{o}{.}\PY{n}{loadtxt}\PY{p}{(}\PY{l+s+s1}{\PYZsq{}}\PY{l+s+s1}{Hz\PYZus{}all.dat}\PY{l+s+s1}{\PYZsq{}}\PY{p}{)} \PY{c+c1}{\PYZsh{}We read the data. They are in the archive Hz\PYZus{}all.dat}
\end{Verbatim}


    \begin{Verbatim}[commandchars=\\\{\}]
{\color{incolor}In [{\color{incolor}113}]:} \PY{l+s+sd}{\PYZsq{}\PYZsq{}\PYZsq{}We save our different data in new variables. }
          \PY{l+s+sd}{z \PYZhy{}\PYZhy{}\PYZhy{}\PYZgt{} Redshift}
          \PY{l+s+sd}{observations \PYZhy{}\PYZhy{}\PYZhy{}\PYZgt{} Our observations}
          \PY{l+s+sd}{errors \PYZhy{}\PYZhy{}\PYZhy{}\PYZgt{} The errors of our observations}
          \PY{l+s+sd}{\PYZsq{}\PYZsq{}\PYZsq{}}
          \PY{n}{z} \PY{o}{=} \PY{n}{dataHz}\PY{p}{[}\PY{p}{:}\PY{p}{,}\PY{l+m+mi}{0}\PY{p}{]}
          \PY{n}{observations} \PY{o}{=} \PY{n}{dataHz}\PY{p}{[}\PY{p}{:}\PY{p}{,}\PY{l+m+mi}{1}\PY{p}{]}
          \PY{n}{errors} \PY{o}{=} \PY{n}{dataHz}\PY{p}{[}\PY{p}{:}\PY{p}{,}\PY{l+m+mi}{2}\PY{p}{]}
\end{Verbatim}


    \begin{Verbatim}[commandchars=\\\{\}]
{\color{incolor}In [{\color{incolor}114}]:} \PY{n}{figsize} \PY{o}{=} \PY{p}{(}\PY{l+m+mi}{8}\PY{p}{,} \PY{l+m+mi}{6}\PY{p}{)}                        \PY{c+c1}{\PYZsh{} definimos el tamaño de nuestra figura}
          \PY{n}{dpi} \PY{o}{=} \PY{l+m+mi}{300}                                \PY{c+c1}{\PYZsh{} dots per inch}
          
          \PY{n}{rcParams}\PY{p}{[}\PY{l+s+s1}{\PYZsq{}}\PY{l+s+s1}{font.size}\PY{l+s+s1}{\PYZsq{}}\PY{p}{]} \PY{o}{=} \PY{l+m+mi}{10}            \PY{c+c1}{\PYZsh{} establecemos el tamaño de la fuente}
          \PY{n}{rcParams}\PY{p}{[}\PY{l+s+s1}{\PYZsq{}}\PY{l+s+s1}{lines.linewidth}\PY{l+s+s1}{\PYZsq{}}\PY{p}{]} \PY{o}{=} \PY{l+m+mi}{1}          \PY{c+c1}{\PYZsh{} el grosor de las líneas}
          \PY{n}{rcParams}\PY{p}{[}\PY{l+s+s1}{\PYZsq{}}\PY{l+s+s1}{mathtext.fontset}\PY{l+s+s1}{\PYZsq{}}\PY{p}{]} \PY{o}{=} \PY{l+s+s1}{\PYZsq{}}\PY{l+s+s1}{cm}\PY{l+s+s1}{\PYZsq{}}      \PY{c+c1}{\PYZsh{} y el tipo de fuente en LaTex }
          
          \PY{n}{fig1} \PY{o}{=} \PY{n}{figure}\PY{p}{(}\PY{n}{figsize}\PY{o}{=}\PY{n}{figsize}\PY{p}{,} \PY{n}{dpi}\PY{o}{=}\PY{n}{dpi}\PY{p}{)}  \PY{c+c1}{\PYZsh{} definimos la figura}
          
           
          \PY{n}{plt}\PY{o}{.}\PY{n}{errorbar}\PY{p}{(}\PY{n}{z}\PY{p}{,} \PY{n}{observations}\PY{p}{,} \PY{n}{errors}\PY{p}{,}      \PY{c+c1}{\PYZsh{}plt.errorbar(x, y, error\PYZus{}y)}
                       \PY{n}{xerr}\PY{o}{=}\PY{n+nb+bp}{None}\PY{p}{,} 
                       \PY{n}{color}\PY{o}{=}\PY{l+s+s1}{\PYZsq{}}\PY{l+s+s1}{red}\PY{l+s+s1}{\PYZsq{}}\PY{p}{,} \PY{n}{marker}\PY{o}{=}\PY{l+s+s1}{\PYZsq{}}\PY{l+s+s1}{o}\PY{l+s+s1}{\PYZsq{}}\PY{p}{,} \PY{n}{ls}\PY{o}{=}\PY{l+s+s1}{\PYZsq{}}\PY{l+s+s1}{None}\PY{l+s+s1}{\PYZsq{}}\PY{p}{,} 
                       \PY{n}{elinewidth} \PY{o}{=}\PY{l+m+mi}{1}\PY{p}{,} \PY{n}{capsize}\PY{o}{=}\PY{l+m+mi}{3}\PY{p}{,} \PY{n}{capthick} \PY{o}{=} \PY{l+m+mi}{1}\PY{p}{,} 
                       \PY{n}{label}\PY{o}{=}\PY{l+s+s1}{\PYZsq{}}\PY{l+s+s1}{\PYZdl{}Datos\PYZdl{}}\PY{l+s+s1}{\PYZsq{}}\PY{p}{)}
          
          
          \PY{n}{plt}\PY{o}{.}\PY{n}{legend}\PY{p}{(}\PY{n}{loc}\PY{o}{=}\PY{l+s+s1}{\PYZsq{}}\PY{l+s+s1}{best}\PY{l+s+s1}{\PYZsq{}}\PY{p}{,} \PY{n}{frameon}\PY{o}{=}\PY{n+nb+bp}{False}\PY{p}{)}     \PY{c+c1}{\PYZsh{}pon la legenda donde mejor quede sin marco}
          
          \PY{n}{xlabel}\PY{p}{(}\PY{l+s+sa}{r}\PY{l+s+s1}{\PYZsq{}}\PY{l+s+s1}{\PYZdl{}z\PYZdl{}}\PY{l+s+s1}{\PYZsq{}}\PY{p}{)}                            
          \PY{n}{ylabel}\PY{p}{(}\PY{l+s+sa}{r}\PY{l+s+s1}{\PYZsq{}}\PY{l+s+s1}{\PYZdl{}H(z) [km/s Mpc\PYZca{}\PYZob{}\PYZhy{}1\PYZcb{}]\PYZdl{}}\PY{l+s+s1}{\PYZsq{}}\PY{p}{)}         
          
          
          \PY{c+c1}{\PYZsh{}savefig(\PYZsq{}data\PYZus{}plot.pdf\PYZsq{}, bbox\PYZus{}inches=\PYZsq{}tight\PYZsq{}) \PYZsh{}guardar la salida como un archivo pdf, eps, jpg, png}
\end{Verbatim}


\begin{Verbatim}[commandchars=\\\{\}]
{\color{outcolor}Out[{\color{outcolor}114}]:} <matplotlib.text.Text at 0x140b6f90>
\end{Verbatim}
            
    \begin{center}
    \adjustimage{max size={0.9\linewidth}{0.9\paperheight}}{Untitled_files/Untitled_4_1.png}
    \end{center}
    { \hspace*{\fill} \\}
    
    \section{Definimos el problema y el modelo
teórico}\label{definimos-el-problema-y-el-modelo-teuxf3rico}

    \subparagraph{\texorpdfstring{Consideremos la extensión más simple del
modelo de LCDM (materia oscura fría con constante cosmológica
\(\Lambda\)). Consideraremos que \(\Lambda\) no es una constante, pero
sigue una ecuación de estado de la forma \(p=\omega\rho\) con
\(\omega\equiv Cte\).}{Consideremos la extensión más simple del modelo de LCDM (materia oscura fría con constante cosmológica \textbackslash{}Lambda). Consideraremos que \textbackslash{}Lambda no es una constante, pero sigue una ecuación de estado de la forma p=\textbackslash{}omega\textbackslash{}rho con \textbackslash{}omega\textbackslash{}equiv Cte.}}\label{consideremos-la-extensiuxf3n-muxe1s-simple-del-modelo-de-lcdm-materia-oscura-fruxeda-con-constante-cosmoluxf3gica-lambda.-consideraremos-que-lambda-no-es-una-constante-pero-sigue-una-ecuaciuxf3n-de-estado-de-la-forma-pomegarho-con-omegaequiv-cte.}

\subparagraph{\texorpdfstring{Si consideramos un universo plano, se
cumple que \(\Omega_m+\Omega_{DE}=1\) \(\Rightarrow\)
\(\Omega_{DE}=1-\Omega_m\)}{Si consideramos un universo plano, se cumple que \textbackslash{}Omega\_m+\textbackslash{}Omega\_\{DE\}=1 \textbackslash{}Rightarrow \textbackslash{}Omega\_\{DE\}=1-\textbackslash{}Omega\_m}}\label{si-consideramos-un-universo-plano-se-cumple-que-omega_momega_de1-rightarrow-omega_de1-omega_m}

\subparagraph{Digamos que utilizamos las "standar candles" (SNIa) como
observaciones para medir a que velocidad se expande el universo para
cierta etapa de la historia del
cosmos.}\label{digamos-que-utilizamos-las-standar-candles-snia-como-observaciones-para-medir-a-que-velocidad-se-expande-el-universo-para-cierta-etapa-de-la-historia-del-cosmos.}

\subparagraph{De cosmología sabemos que el parámetro de Hubble
evoluciona en función del Redshift y el contenido de la materia
como}\label{de-cosmologuxeda-sabemos-que-el-paruxe1metro-de-hubble-evoluciona-en-funciuxf3n-del-redshift-y-el-contenido-de-la-materia-como}

\subsubsection{\texorpdfstring{\(H(z)=H_0\sqrt{\Omega_m(1+z)^3+\Omega_\Lambda}\)}{H(z)=H\_0\textbackslash{}sqrt\{\textbackslash{}Omega\_m(1+z)\^{}3+\textbackslash{}Omega\_\textbackslash{}Lambda\}}}\label{hzh_0sqrtomega_m1z3omega_lambda}

\subparagraph{\texorpdfstring{donde \(H_0\) es el valor del parámetro de
Huble en nuestra
época.}{donde H\_0 es el valor del parámetro de Huble en nuestra época.}}\label{donde-h_0-es-el-valor-del-paruxe1metro-de-huble-en-nuestra-uxe9poca.}

Los parámetros que vamos a estimar serán el valor de \(H_0\), \(\omega\)
y \(\Omega_m\).

    \begin{Verbatim}[commandchars=\\\{\}]
{\color{incolor}In [{\color{incolor}115}]:} \PY{k}{def} \PY{n+nf}{hubblefunc}\PY{p}{(}\PY{n}{z}\PY{p}{,} \PY{n}{w}\PY{p}{,} \PY{n}{H0}\PY{p}{,} \PY{n}{OmegaM}\PY{p}{)}\PY{p}{:}
              \PY{l+s+sd}{\PYZsq{}\PYZsq{}\PYZsq{}}
          \PY{l+s+sd}{    This function calculates the theoretical value for the Hubble function,}
          \PY{l+s+sd}{    this is, the expansion rate of the Universe in terms of redshift}
          \PY{l+s+sd}{    }
          \PY{l+s+sd}{    w: Dark Energy Equation of State. w=\PYZhy{}1 for a Cosmological Constant}
          \PY{l+s+sd}{    H0: present value of Hubble constant}
          \PY{l+s+sd}{    OmegaM: fractional matter density}
          \PY{l+s+sd}{    }
          \PY{l+s+sd}{    \PYZsq{}\PYZsq{}\PYZsq{}}
              \PY{n}{matter\PYZus{}contribution} \PY{o}{=} \PY{n}{OmegaM} \PY{o}{*}\PY{p}{(}\PY{l+m+mi}{1} \PY{o}{+} \PY{n}{z}\PY{p}{)}\PY{o}{*}\PY{o}{*}\PY{l+m+mi}{3}
              
              \PY{n}{DE\PYZus{}contribution} \PY{o}{=} \PY{p}{(}\PY{l+m+mi}{1} \PY{o}{\PYZhy{}} \PY{n}{OmegaM}\PY{p}{)} \PY{o}{*} \PY{p}{(}\PY{l+m+mi}{1} \PY{o}{+} \PY{n}{z}\PY{p}{)}\PY{o}{*}\PY{o}{*}\PY{p}{(}\PY{l+m+mi}{3} \PY{o}{*} \PY{p}{(}\PY{l+m+mi}{1} \PY{o}{+} \PY{n}{w}\PY{p}{)}\PY{p}{)}
          
              \PY{n}{Ez} \PY{o}{=} \PY{n}{np}\PY{o}{.}\PY{n}{sqrt}\PY{p}{(}\PY{n}{matter\PYZus{}contribution} \PY{o}{+} \PY{n}{DE\PYZus{}contribution}\PY{p}{)}
              
              \PY{k}{return} \PY{n}{H0}\PY{o}{*}\PY{n}{Ez}
\end{Verbatim}


    \section{Hacemos la inferencia de
parámetros}\label{hacemos-la-inferencia-de-paruxe1metros}

    \subparagraph{Vamos a considerar un Likelihood Gaussiano,
i.e.}\label{vamos-a-considerar-un-likelihood-gaussiano-i.e.}

\(L(\vec\theta)=exp\left[-\frac{1}{2}\sum_i^{No. Obs}\frac{(H(z_i)_{obs}-H(z_i,w=-1,H_0,\Omega_m))^2}{\sigma_i^2}\right]\)
\#\#\#\#\# donde \(\vec\theta=\{H_0,M\}\). \#\#\#\#\# Vamos a suponer
que no tenemos información de los parámetros, salvo sus cotas límite.
Digamos que \(\Omega_m\in [0.1,1]\) y \(H_0\in [10,100]\). Por la
ignorancia de nuestros parámetros, un buen prior que podemos considerar
para estos es tomar un prior uniforme \#\#\#\#\#
\(\Omega_m\sim U[0.1,1]\) \#\#\#\#\# \(H_0\sim U[10,100]\)

    \begin{Verbatim}[commandchars=\\\{\}]
{\color{incolor}In [{\color{incolor}116}]:} \PY{l+s+sd}{\PYZsq{}\PYZsq{}\PYZsq{}Now we are going to give our model to PyMC\PYZsq{}\PYZsq{}\PYZsq{}}
          \PY{k}{with} \PY{n}{pm}\PY{o}{.}\PY{n}{Model}\PY{p}{(}\PY{p}{)} \PY{k}{as} \PY{n}{model}\PY{p}{:} 
              \PY{n}{OmegaM} \PY{o}{=} \PY{n}{pm}\PY{o}{.}\PY{n}{Uniform}\PY{p}{(}\PY{l+s+s1}{\PYZsq{}}\PY{l+s+s1}{OmegaM}\PY{l+s+s1}{\PYZsq{}}\PY{p}{,}\PY{n}{lower}\PY{o}{=}\PY{l+m+mf}{0.1}\PY{p}{,}\PY{n}{upper}\PY{o}{=}\PY{l+m+mf}{1.0}\PY{p}{)}
              \PY{n}{H0} \PY{o}{=} \PY{n}{pm}\PY{o}{.}\PY{n}{Uniform}\PY{p}{(}\PY{l+s+s1}{\PYZsq{}}\PY{l+s+s1}{H0}\PY{l+s+s1}{\PYZsq{}}\PY{p}{,}\PY{n}{lower}\PY{o}{=}\PY{l+m+mf}{10.0}\PY{p}{,}\PY{n}{upper}\PY{o}{=}\PY{l+m+mf}{100.0}\PY{p}{)}
              \PY{n}{H\PYZus{}z} \PY{o}{=} \PY{n}{hubblefunc}\PY{p}{(}\PY{n}{z}\PY{p}{,} \PY{o}{\PYZhy{}}\PY{l+m+mi}{1}\PY{p}{,} \PY{n}{H0}\PY{p}{,} \PY{n}{OmegaM}\PY{p}{)}
          
              \PY{n}{Lik} \PY{o}{=} \PY{n}{pm}\PY{o}{.}\PY{n}{Normal}\PY{p}{(}\PY{l+s+s1}{\PYZsq{}}\PY{l+s+s1}{Lik}\PY{l+s+s1}{\PYZsq{}}\PY{p}{,}\PY{n}{mu}\PY{o}{=}\PY{n}{H\PYZus{}z}\PY{p}{,}\PY{n}{sd}\PY{o}{=}\PY{p}{(}\PY{l+m+mi}{2}\PY{o}{*}\PY{o}{*}\PY{p}{(}\PY{l+m+mi}{1}\PY{o}{/}\PY{l+m+mi}{2}\PY{p}{)}\PY{p}{)}\PY{o}{*}\PY{n}{errors}\PY{p}{,}\PY{n}{observed}\PY{o}{=}\PY{n}{observations}\PY{p}{)}
              
\end{Verbatim}


    \begin{Verbatim}[commandchars=\\\{\}]
{\color{incolor}In [{\color{incolor}117}]:} \PY{c+c1}{\PYZsh{}We specify the number of iterations}
          \PY{n}{niter}\PY{o}{=}\PY{l+m+mi}{50000}
          \PY{k}{with} \PY{n}{model}\PY{p}{:}
              \PY{n}{start} \PY{o}{=} \PY{n}{pm}\PY{o}{.}\PY{n}{find\PYZus{}MAP}\PY{p}{(}\PY{p}{)}
              \PY{n}{step} \PY{o}{=} \PY{n}{pm}\PY{o}{.}\PY{n}{Metropolis}\PY{p}{(}\PY{p}{)}
              \PY{n}{db} \PY{o}{=} \PY{n}{SQLite}\PY{p}{(}\PY{l+s+s1}{\PYZsq{}}\PY{l+s+s1}{trace.db}\PY{l+s+s1}{\PYZsq{}}\PY{p}{)}
              \PY{n}{trace} \PY{o}{=} \PY{n}{pm}\PY{o}{.}\PY{n}{sample}\PY{p}{(}\PY{n}{niter}\PY{p}{,} \PY{n}{trace}\PY{o}{=}\PY{n}{db}\PY{p}{,} \PY{n}{step}\PY{o}{=}\PY{n}{step}\PY{p}{,} \PY{n}{start}\PY{o}{=}\PY{n}{start}\PY{p}{,}\PY{n}{njobs}\PY{o}{=}\PY{l+m+mi}{5}\PY{p}{,} \PY{n}{random\PYZus{}seed}\PY{o}{=}\PY{l+m+mi}{123}\PY{p}{)}
\end{Verbatim}


    \begin{Verbatim}[commandchars=\\\{\}]
{\color{incolor}In [{\color{incolor}118}]:} \PY{n}{start}
\end{Verbatim}


\begin{Verbatim}[commandchars=\\\{\}]
{\color{outcolor}Out[{\color{outcolor}118}]:} \{'H0': array(68.21080972),
           'H0\_interval\_\_': array(0.60494477),
           'OmegaM': array(0.31849471),
           'OmegaM\_interval\_\_': array(-1.13754222)\}
\end{Verbatim}
            
    \begin{Verbatim}[commandchars=\\\{\}]
{\color{incolor}In [{\color{incolor}119}]:} \PY{k}{with} \PY{n}{model}\PY{p}{:}
              \PY{n}{tracee} \PY{o}{=} \PY{n}{pm}\PY{o}{.}\PY{n}{backends}\PY{o}{.}\PY{n}{sqlite}\PY{o}{.}\PY{n}{load}\PY{p}{(}\PY{l+s+s1}{\PYZsq{}}\PY{l+s+s1}{trace.db}\PY{l+s+s1}{\PYZsq{}}\PY{p}{)}
              \PY{n}{pm}\PY{o}{.}\PY{n}{traceplot}\PY{p}{(}\PY{n}{tracee}\PY{p}{,} \PY{n}{varnames}\PY{o}{=}\PY{p}{[}\PY{l+s+s1}{\PYZsq{}}\PY{l+s+s1}{OmegaM}\PY{l+s+s1}{\PYZsq{}}\PY{p}{,}\PY{l+s+s1}{\PYZsq{}}\PY{l+s+s1}{H0}\PY{l+s+s1}{\PYZsq{}}\PY{p}{]}\PY{p}{)}
\end{Verbatim}


    \begin{center}
    \adjustimage{max size={0.9\linewidth}{0.9\paperheight}}{Untitled_files/Untitled_13_0.png}
    \end{center}
    { \hspace*{\fill} \\}
    
    \begin{Verbatim}[commandchars=\\\{\}]
{\color{incolor}In [{\color{incolor}139}]:} \PY{n}{t} \PY{o}{=} \PY{n}{trace}\PY{p}{[}\PY{n}{niter}\PY{o}{/}\PY{o}{/}\PY{l+m+mi}{2}\PY{p}{:}\PY{p}{]}
          \PY{n}{t}\PY{p}{[}\PY{l+s+s1}{\PYZsq{}}\PY{l+s+s1}{OmegaM}\PY{l+s+s1}{\PYZsq{}}\PY{p}{]}\PY{o}{.}\PY{n}{shape}
          \PY{n}{t}\PY{p}{[}\PY{l+s+s1}{\PYZsq{}}\PY{l+s+s1}{H0}\PY{l+s+s1}{\PYZsq{}}\PY{p}{]}\PY{o}{.}\PY{n}{shape}
\end{Verbatim}


\begin{Verbatim}[commandchars=\\\{\}]
{\color{outcolor}Out[{\color{outcolor}139}]:} (1075000,)
\end{Verbatim}
            
    \begin{Verbatim}[commandchars=\\\{\}]
{\color{incolor}In [{\color{incolor}121}]:} \PY{n}{OmegaM} \PY{o}{=} \PY{n}{trace}\PY{o}{.}\PY{n}{get\PYZus{}values}\PY{p}{(}\PY{l+s+s1}{\PYZsq{}}\PY{l+s+s1}{OmegaM}\PY{l+s+s1}{\PYZsq{}}\PY{p}{,} \PY{n}{burn}\PY{o}{=}\PY{n}{niter}\PY{o}{/}\PY{o}{/}\PY{l+m+mi}{2}\PY{p}{,} \PY{n}{combine}\PY{o}{=}\PY{n+nb+bp}{True}\PY{p}{,} \PY{n}{chains}\PY{o}{=}\PY{p}{[}\PY{l+m+mi}{0}\PY{p}{,}\PY{l+m+mi}{2}\PY{p}{]}\PY{p}{)}
          \PY{n}{OmegaM}\PY{o}{.}\PY{n}{shape}
          
          \PY{n}{H0} \PY{o}{=} \PY{n}{trace}\PY{o}{.}\PY{n}{get\PYZus{}values}\PY{p}{(}\PY{l+s+s1}{\PYZsq{}}\PY{l+s+s1}{H0}\PY{l+s+s1}{\PYZsq{}}\PY{p}{,} \PY{n}{burn}\PY{o}{=}\PY{n}{niter}\PY{o}{/}\PY{o}{/}\PY{l+m+mi}{2}\PY{p}{,} \PY{n}{combine}\PY{o}{=}\PY{n+nb+bp}{True}\PY{p}{,} \PY{n}{chains}\PY{o}{=}\PY{p}{[}\PY{l+m+mi}{0}\PY{p}{,}\PY{l+m+mi}{2}\PY{p}{]}\PY{p}{)}
          \PY{n}{H0}\PY{o}{.}\PY{n}{shape}
\end{Verbatim}


\begin{Verbatim}[commandchars=\\\{\}]
{\color{outcolor}Out[{\color{outcolor}121}]:} (430000,)
\end{Verbatim}
            
    \begin{Verbatim}[commandchars=\\\{\}]
{\color{incolor}In [{\color{incolor}122}]:} \PY{n}{pm}\PY{o}{.}\PY{n}{autocorrplot}\PY{p}{(}\PY{n}{t}\PY{p}{,} \PY{n}{varnames}\PY{o}{=}\PY{p}{[}\PY{l+s+s1}{\PYZsq{}}\PY{l+s+s1}{OmegaM}\PY{l+s+s1}{\PYZsq{}}\PY{p}{,}\PY{l+s+s1}{\PYZsq{}}\PY{l+s+s1}{H0}\PY{l+s+s1}{\PYZsq{}}\PY{p}{]}\PY{p}{)}
          \PY{k}{pass}
\end{Verbatim}


    \begin{center}
    \adjustimage{max size={0.9\linewidth}{0.9\paperheight}}{Untitled_files/Untitled_16_0.png}
    \end{center}
    { \hspace*{\fill} \\}
    
    \begin{Verbatim}[commandchars=\\\{\}]
{\color{incolor}In [{\color{incolor}123}]:} \PY{n}{pm}\PY{o}{.}\PY{n}{effective\PYZus{}n}\PY{p}{(}\PY{n}{t}\PY{p}{)}
\end{Verbatim}


\begin{Verbatim}[commandchars=\\\{\}]
{\color{outcolor}Out[{\color{outcolor}123}]:} \{'H0': 34391.0, 'OmegaM': 7537.0\}
\end{Verbatim}
            
    \begin{Verbatim}[commandchars=\\\{\}]
{\color{incolor}In [{\color{incolor}124}]:} \PY{n}{pm}\PY{o}{.}\PY{n}{gelman\PYZus{}rubin}\PY{p}{(}\PY{n}{t}\PY{p}{)}
\end{Verbatim}


\begin{Verbatim}[commandchars=\\\{\}]
{\color{outcolor}Out[{\color{outcolor}124}]:} \{'H0': 1.0004853292008946, 'OmegaM': 1.0000401305986388\}
\end{Verbatim}
            
    \begin{Verbatim}[commandchars=\\\{\}]
{\color{incolor}In [{\color{incolor}125}]:} \PY{n}{plt}\PY{o}{.}\PY{n}{plot}\PY{p}{(}\PY{n}{pm}\PY{o}{.}\PY{n}{geweke}\PY{p}{(}\PY{n}{t}\PY{p}{[}\PY{l+s+s1}{\PYZsq{}}\PY{l+s+s1}{OmegaM}\PY{l+s+s1}{\PYZsq{}}\PY{p}{]}\PY{p}{)}\PY{p}{[}\PY{p}{:}\PY{p}{,}\PY{l+m+mi}{1}\PY{p}{]}\PY{p}{,} \PY{l+s+s1}{\PYZsq{}}\PY{l+s+s1}{o}\PY{l+s+s1}{\PYZsq{}}\PY{p}{)}
          \PY{n}{plt}\PY{o}{.}\PY{n}{axhline}\PY{p}{(}\PY{l+m+mi}{1}\PY{p}{,} \PY{n}{c}\PY{o}{=}\PY{l+s+s1}{\PYZsq{}}\PY{l+s+s1}{red}\PY{l+s+s1}{\PYZsq{}}\PY{p}{)}
          \PY{n}{plt}\PY{o}{.}\PY{n}{axhline}\PY{p}{(}\PY{o}{\PYZhy{}}\PY{l+m+mi}{1}\PY{p}{,} \PY{n}{c}\PY{o}{=}\PY{l+s+s1}{\PYZsq{}}\PY{l+s+s1}{red}\PY{l+s+s1}{\PYZsq{}}\PY{p}{)}
          \PY{n}{plt}\PY{o}{.}\PY{n}{gca}\PY{p}{(}\PY{p}{)}\PY{o}{.}\PY{n}{margins}\PY{p}{(}\PY{l+m+mf}{0.05}\PY{p}{)}
          \PY{k}{pass}
\end{Verbatim}


    \begin{center}
    \adjustimage{max size={0.9\linewidth}{0.9\paperheight}}{Untitled_files/Untitled_19_0.png}
    \end{center}
    { \hspace*{\fill} \\}
    
    \begin{Verbatim}[commandchars=\\\{\}]
{\color{incolor}In [{\color{incolor}126}]:} \PY{n}{plt}\PY{o}{.}\PY{n}{plot}\PY{p}{(}\PY{n}{pm}\PY{o}{.}\PY{n}{geweke}\PY{p}{(}\PY{n}{t}\PY{p}{[}\PY{l+s+s1}{\PYZsq{}}\PY{l+s+s1}{H0}\PY{l+s+s1}{\PYZsq{}}\PY{p}{]}\PY{p}{)}\PY{p}{[}\PY{p}{:}\PY{p}{,}\PY{l+m+mi}{1}\PY{p}{]}\PY{p}{,} \PY{l+s+s1}{\PYZsq{}}\PY{l+s+s1}{o}\PY{l+s+s1}{\PYZsq{}}\PY{p}{)}
          \PY{n}{plt}\PY{o}{.}\PY{n}{axhline}\PY{p}{(}\PY{l+m+mi}{1}\PY{p}{,} \PY{n}{c}\PY{o}{=}\PY{l+s+s1}{\PYZsq{}}\PY{l+s+s1}{red}\PY{l+s+s1}{\PYZsq{}}\PY{p}{)}
          \PY{n}{plt}\PY{o}{.}\PY{n}{axhline}\PY{p}{(}\PY{o}{\PYZhy{}}\PY{l+m+mi}{1}\PY{p}{,} \PY{n}{c}\PY{o}{=}\PY{l+s+s1}{\PYZsq{}}\PY{l+s+s1}{red}\PY{l+s+s1}{\PYZsq{}}\PY{p}{)}
          \PY{n}{plt}\PY{o}{.}\PY{n}{gca}\PY{p}{(}\PY{p}{)}\PY{o}{.}\PY{n}{margins}\PY{p}{(}\PY{l+m+mf}{0.05}\PY{p}{)}
          \PY{k}{pass}
\end{Verbatim}


    \begin{center}
    \adjustimage{max size={0.9\linewidth}{0.9\paperheight}}{Untitled_files/Untitled_20_0.png}
    \end{center}
    { \hspace*{\fill} \\}
    
    \begin{Verbatim}[commandchars=\\\{\}]
{\color{incolor}In [{\color{incolor}127}]:} \PY{n}{pm}\PY{o}{.}\PY{n}{summary}\PY{p}{(}\PY{n}{t}\PY{p}{,} \PY{n}{varnames}\PY{o}{=}\PY{p}{[}\PY{l+s+s1}{\PYZsq{}}\PY{l+s+s1}{OmegaM}\PY{l+s+s1}{\PYZsq{}}\PY{p}{,}\PY{l+s+s1}{\PYZsq{}}\PY{l+s+s1}{H0}\PY{l+s+s1}{\PYZsq{}}\PY{p}{]}\PY{p}{)}
\end{Verbatim}


\begin{Verbatim}[commandchars=\\\{\}]
{\color{outcolor}Out[{\color{outcolor}127}]:}              mean        sd  mc\_error    hpd\_2.5   hpd\_97.5    n\_eff      Rhat
          OmegaM   0.317201  0.072718  0.001001   0.164740   0.469017   7537.0  1.000040
          H0      67.691635  4.201709  0.025852  59.541644  76.058639  34391.0  1.000485
\end{Verbatim}
            
    \begin{Verbatim}[commandchars=\\\{\}]
{\color{incolor}In [{\color{incolor}128}]:} \PY{n}{pm}\PY{o}{.}\PY{n}{traceplot}\PY{p}{(}\PY{n}{t}\PY{p}{,} \PY{n}{varnames}\PY{o}{=}\PY{p}{[}\PY{l+s+s1}{\PYZsq{}}\PY{l+s+s1}{OmegaM}\PY{l+s+s1}{\PYZsq{}}\PY{p}{,}\PY{l+s+s1}{\PYZsq{}}\PY{l+s+s1}{H0}\PY{l+s+s1}{\PYZsq{}}\PY{p}{]}\PY{p}{)}
          \PY{k}{pass}
\end{Verbatim}


    \begin{center}
    \adjustimage{max size={0.9\linewidth}{0.9\paperheight}}{Untitled_files/Untitled_22_0.png}
    \end{center}
    { \hspace*{\fill} \\}
    
    \begin{Verbatim}[commandchars=\\\{\}]
{\color{incolor}In [{\color{incolor}144}]:} \PY{n}{df\PYZus{}trace} \PY{o}{=} \PY{n}{pm}\PY{o}{.}\PY{n}{trace\PYZus{}to\PYZus{}dataframe}\PY{p}{(}\PY{n}{trace}\PY{p}{[}\PY{n}{niter}\PY{o}{/}\PY{o}{/}\PY{l+m+mi}{2}\PY{p}{:}\PY{p}{]}\PY{p}{)}
          \PY{n}{pd}\PY{o}{.}\PY{n}{scatter\PYZus{}matrix}\PY{p}{(}\PY{n}{df\PYZus{}trace}\PY{o}{.}\PY{n}{ix}\PY{p}{[}\PY{o}{\PYZhy{}}\PY{n}{niter}\PY{o}{/}\PY{o}{/}\PY{l+m+mi}{2}\PY{p}{:}\PY{p}{,} \PY{p}{[}\PY{l+s+s1}{\PYZsq{}}\PY{l+s+s1}{OmegaM}\PY{l+s+s1}{\PYZsq{}}\PY{p}{,} \PY{l+s+s1}{\PYZsq{}}\PY{l+s+s1}{H0}\PY{l+s+s1}{\PYZsq{}}\PY{p}{]}\PY{p}{]}\PY{p}{,} \PY{n}{diagonal}\PY{o}{=}\PY{l+s+s1}{\PYZsq{}}\PY{l+s+s1}{kde}\PY{l+s+s1}{\PYZsq{}}\PY{p}{)}
          \PY{n}{plt}\PY{o}{.}\PY{n}{tight\PYZus{}layout}\PY{p}{(}\PY{p}{)}
          \PY{n}{plt}\PY{o}{.}\PY{n}{show}\PY{p}{(}\PY{p}{)}
          \PY{k}{pass}
\end{Verbatim}


    \begin{Verbatim}[commandchars=\\\{\}]
/home/enriques/Downloads/yes/lib/python2.7/site-packages/ipykernel/\_\_main\_\_.py:2: DeprecationWarning: 
.ix is deprecated. Please use
.loc for label based indexing or
.iloc for positional indexing

See the documentation here:
http://pandas.pydata.org/pandas-docs/stable/indexing.html\#ix-indexer-is-deprecated
  from ipykernel import kernelapp as app
/home/enriques/Downloads/yes/lib/python2.7/site-packages/ipykernel/\_\_main\_\_.py:2: FutureWarning: pandas.scatter\_matrix is deprecated. Use pandas.plotting.scatter\_matrix instead
  from ipykernel import kernelapp as app

    \end{Verbatim}

    \begin{center}
    \adjustimage{max size={0.9\linewidth}{0.9\paperheight}}{Untitled_files/Untitled_23_1.png}
    \end{center}
    { \hspace*{\fill} \\}
    
    \begin{Verbatim}[commandchars=\\\{\}]
{\color{incolor}In [{\color{incolor}165}]:} \PY{n}{plot}\PY{p}{(}\PY{n}{OmegaM}\PY{p}{,} \PY{n}{H0}\PY{p}{,}
               \PY{n}{linestyle}\PY{o}{=}\PY{l+s+s1}{\PYZsq{}}\PY{l+s+s1}{none}\PY{l+s+s1}{\PYZsq{}}\PY{p}{,} \PY{n}{marker}\PY{o}{=}\PY{l+s+s1}{\PYZsq{}}\PY{l+s+s1}{o}\PY{l+s+s1}{\PYZsq{}}\PY{p}{,} \PY{n}{color}\PY{o}{=}\PY{l+s+s1}{\PYZsq{}}\PY{l+s+s1}{grey}\PY{l+s+s1}{\PYZsq{}}\PY{p}{,} \PY{n}{mec}\PY{o}{=}\PY{l+s+s1}{\PYZsq{}}\PY{l+s+s1}{grey}\PY{l+s+s1}{\PYZsq{}}\PY{p}{,}
          \PY{n}{alpha}\PY{o}{=}\PY{o}{.}\PY{l+m+mo}{05}\PY{p}{,} \PY{n}{label}\PY{o}{=}\PY{l+s+s1}{\PYZsq{}}\PY{l+s+s1}{Posterior}\PY{l+s+s1}{\PYZsq{}}\PY{p}{,} \PY{n}{zorder}\PY{o}{=}\PY{o}{\PYZhy{}}\PY{l+m+mi}{100}\PY{p}{)}
          
          \PY{k+kn}{import} \PY{n+nn}{scipy.stats}
          \PY{n}{gkde} \PY{o}{=} \PY{n}{scipy}\PY{o}{.}\PY{n}{stats}\PY{o}{.}\PY{n}{gaussian\PYZus{}kde}\PY{p}{(}\PY{p}{[}\PY{n}{OmegaM}\PY{p}{,} \PY{n}{H0}\PY{p}{]}\PY{p}{)}
          \PY{n}{x}\PY{p}{,}\PY{n}{y} \PY{o}{=} \PY{n}{mgrid}\PY{p}{[}\PY{l+m+mf}{0.}\PY{p}{:}\PY{l+m+mf}{1.}\PY{p}{:}\PY{o}{.}\PY{l+m+mo}{005}\PY{p}{,} \PY{l+m+mf}{10.}\PY{p}{:}\PY{l+m+mf}{100.}\PY{p}{:}\PY{o}{.}\PY{l+m+mi}{5}\PY{p}{]}
          \PY{n}{z} \PY{o}{=} \PY{n}{array}\PY{p}{(}\PY{n}{gkde}\PY{o}{.}\PY{n}{evaluate}\PY{p}{(}\PY{p}{[}\PY{n}{x}\PY{o}{.}\PY{n}{flatten}\PY{p}{(}\PY{p}{)}\PY{p}{,}\PY{n}{y}\PY{o}{.}\PY{n}{flatten}\PY{p}{(}\PY{p}{)}\PY{p}{]}\PY{p}{)}\PY{p}{)}\PY{o}{.}\PY{n}{reshape}\PY{p}{(}\PY{n}{x}\PY{o}{.}\PY{n}{shape}\PY{p}{)}
          \PY{n}{contourf}\PY{p}{(}\PY{n}{x}\PY{p}{,} \PY{n}{y}\PY{p}{,} \PY{n}{z}\PY{p}{,}\PY{l+m+mi}{3}\PY{p}{,} \PY{n}{linewidths}\PY{o}{=}\PY{l+m+mi}{1}\PY{p}{,} \PY{n}{alpha}\PY{o}{=}\PY{o}{.}\PY{l+m+mi}{5}\PY{p}{,} \PY{n}{cmap}\PY{o}{=}\PY{l+s+s1}{\PYZsq{}}\PY{l+s+s1}{Greys}\PY{l+s+s1}{\PYZsq{}}\PY{p}{)}
          \PY{n}{plt}\PY{o}{.}\PY{n}{colorbar}\PY{p}{(}\PY{p}{)}\PY{p}{;}
          
          \PY{n}{ylabel}\PY{p}{(}\PY{l+s+sa}{r}\PY{l+s+s1}{\PYZsq{}}\PY{l+s+s1}{\PYZdl{}H0\PYZdl{}}\PY{l+s+s1}{\PYZsq{}}\PY{p}{,} \PY{n}{fontsize}\PY{o}{=}\PY{l+m+mi}{18}\PY{p}{,} \PY{n}{rotation}\PY{o}{=}\PY{l+m+mi}{0}\PY{p}{)}
          \PY{n}{xlabel}\PY{p}{(}\PY{l+s+sa}{r}\PY{l+s+s1}{\PYZsq{}}\PY{l+s+s1}{\PYZdl{}}\PY{l+s+s1}{\PYZbs{}}\PY{l+s+s1}{Omega\PYZus{}m\PYZdl{}}\PY{l+s+s1}{\PYZsq{}}\PY{p}{,} \PY{n}{fontsize}\PY{o}{=}\PY{l+m+mi}{18}\PY{p}{)}
          \PY{n}{legend}\PY{p}{(}\PY{p}{)}
          \PY{n}{axis}\PY{p}{(}\PY{p}{[}\PY{l+m+mf}{0.2}\PY{p}{,} \PY{l+m+mf}{0.6}\PY{p}{,} \PY{l+m+mf}{60.}\PY{p}{,} \PY{l+m+mf}{80.}\PY{p}{]}\PY{p}{)}
          \PY{n}{savefig}\PY{p}{(}\PY{l+s+s1}{\PYZsq{}}\PY{l+s+s1}{param\PYZus{}dist.png}\PY{l+s+s1}{\PYZsq{}}\PY{p}{)}
\end{Verbatim}


    \begin{center}
    \adjustimage{max size={0.9\linewidth}{0.9\paperheight}}{Untitled_files/Untitled_24_0.png}
    \end{center}
    { \hspace*{\fill} \\}
    

    % Add a bibliography block to the postdoc
    
    
    
    \end{document}
